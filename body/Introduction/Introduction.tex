% !TEX root = ../../thesis.tex


\newpage
\vphantom{}
%the * is to not number the section
\section*{Introduction}
    Waves are ubiquitous in nature. Given a stable medium, in the sense that it possesses a minimum energy configuration that cannot be broken by a small perturbation, it is a certainty that some kind of oscillatory phenomenon takes place. In a fluid local velocity variations generate pressure waves. On the boundary layer separating two fluids (water and air for example) surface waves oscillate. In a rigid body, tightly bound atoms vibrate in collective sound waves. One does not even need to consider a medium in the Orthodox sense, as accelerating charges produce synchronized oscillation of the electric and magnetic field, propagating into the vacuum. And again one should not restrict one self to the classical world, as in an ensemble of superfluid particles local variations of density travel macroscopically in an oscillatory fashion. \\
    However, the full phenomenology of this and many others physical settings can be rarely completely captured considering their waves as free, i.e. assuming they obey the D'Alambert equation. The motion in time of a single harmonic of a given wavelength is not oblivious to the motion of all others, different waves interact and influence each other. This complexity is analyzed through the tools of nonlinear physics, as the true governing equations of a great deal of physical systems are in fact nonlinear.\\
    The inherent difficulty of such equations, coupled to the practical impossibility of experimentally gathering initial conditions, led to a statistical approach in the spirit of Boltzmann. The first quantum kinetic equation was derived in 1929 by Peierls \cite{Peierls1929}. Its goal was to describe the kinetic of phonons in an anharmonic crystal. The focus of this early applications, however, was on systems very close to thermodynamic equilibrium, without considering possible exchanges of energy with the environment\footnote{Here the word environment means any physical phenomenon not described by the theory under analysis, for example the atmosphere for the ocean.}. The Peierls equation, in the limit of low phonon density, unsurprisingly reduces to the Boltzmann equation. In the limit of high phonon density, it becomes a kinetic equation with a continuous infinity of degrees of freedom, now known as the wave kinetic equation.\\
    It was only in the 1960s that interest in turbulent states in plasmas and geophysical planetary physics led to the derivation of the first wave kinetic equation, by Hassleman in 1962 \cite{Hasselmann1962}. In 1965 Zakharov \cite{Zakharov1965} found analytical solutions describing out of equilibrium stationary states, soon generalized to a wide variety of physical systems. These solutions describe situations in which conserved quantities are transferred in cascades from large to small scales or vice versa, enabling, for example, the modelization of  turbulent plasma in tokamaks or energy transfer from the wind to the ocean surface. They are called Kolmogorv-Zakharov (KZ) solutions, in analogy with the spectrum of hydrodynamic turbulence discovered by Kolmogorov in 1941 \cite{1991}. The study of this out-of-equilibrium solutions constitutes the core of weak turbulence theory (WT), named after the perturbative assumptions needed to derive the WKE. \\ 
    Nowadays, the applications of WT spans across various fields of physics. From capillary, surface, internal and inertial waves in oceanic and atmospheric turbulence to superfluid and Bose-Einstein condensates turbulence. For a thorough account of modern applications see \cite{Nazarenko2011}.\\ 
    %\thispagestyle{intro2}
    There are multiple ways to derive the WKE for a given system, what they all share is the need for some kind of perturbative expansion. Usually this expansion is truncated to the first nontrivial term. In the last years a novel approach was developed in \cite{Rosenhaus:2022uwa} \cite{Rosenhaus:2023pdj} \cite{Rosenhaus:2023sik} \cite{Rosenhaus2023}, that allows for more straightforward calculations of higher order contributions. Our thesis stems from these works. \\

    In the \textbf{first chapter} we introduce the Hamiltonian formulation of a generic nonlinear equation of the four-wave kind\footnote{Loosely meaning that the nonlinearity is present in the differential equation only as a third power of the amplitudes.}. We then review the derivation, presented in \cite{Onorato2020}, of the WKE for this specific class of systems. It starts with the standard perturbation theory of a nonlinear differential equation, and later adds statistical assumptions over the initial distributions of phases and amplitudes. We then analyze the main properties and conservation laws of the kinetic equation thus found. During the discussion a notion of entropy is introduced to prove an H-theorem, implying irreversibility. In the end, we discuss the different types of stationary solutions. From the maximization of entropy, we find the Rayleigh-Jeans solution, also called the thermal stationary state. Following dimensional arguments, we find another set of spectra, known as the Kolmogorov-Zakharov solutions. These spectra are characterized by nonzero fluxes of conserved quantities, we consequently discuss the coupling of the kinetic equation to the environment to allow for such fluxes.\\
    In the \textbf{second chapter} we review the path integral approach to a weakly nonlinear system, introduced by Rosenhaus and Smolkin in \cite{Rosenhaus2023}. The basic idea consists in inserting nonlocalized forcing and dissipative terms in the nonlinear microscopic equations. One can then impose Gaussian statistics over the forcing to introduce randomness into the system. The statistical theory can be moved, thanks to the path integral formalism, from the forcing to the amplitudes, resulting in a Euclidean field theory. There one can utilize the well-developed perturbation theory of quantum field theory (QFT) to perform otherwise extremely cumbersome calculations. After presenting this new method, we discuss the Feynman rules of the theory and evaluate the first two orders of the two and four points correlators. Thanks to those we are able to write a higher order WKE, that we briefly discuss.\\
    %\thispagestyle{intro2}
    In the \textbf{third chapter} we start by describing a family of one dimensional models, called MMT from the names of Majda, McLaughlin and Tabak. They were originally introduced, in \cite{Majda1997}, as a relatively simple case study for WT, where fine grid numerical simulations were not as expensive as in higher dimensional systems. We then study the aforementioned higher order corrections for the MMT model with dispersion relation $\omga (k) = |k|^\frac{1}{2}$, to mimic water waves. After proving that the expected conservation laws holds for this new WKE, we uncover a set of divergencies undermining the validity of this higher order corrections. We conclude the chapter with numerical simulations of the leading order kinetic equation, using the solver WavKins developed by Krstulovic and Labarre. Our results confirm the predicted  KZ spectra and their stability. 



    
%\newpage
%\thispagestyle{intro2}
%provaprova

