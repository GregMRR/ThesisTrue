% !TEX root = ../../thesis.tex


\newpage
\vphantom{}
%the * is to not number the section
\section*{Introduction}
    
    Waves are ubuquitus in nature. Given a stable medium, in the sense that it possesses a minimum energy configuration that cannot be broken by a small perturbation, it is a certainty that some kind of oscillatory phenomenon takes place. We could think about a fluid, the most common state of matter, where the relative position of molecules variates, originating pressure waves. We could think about its separation layer from another one, where surface tension tries to minimize the area, generating surface waves. We could think about a rigid body, with tightly bound atoms vibrating in collective sound waves. One does not even need to consider a medium in the classic sense, as accelerating charges produce synchronized oscillation of the electric and magnetic field, propagating into the vacuum. And again one should not restrict oneselves to the classical world, when local variations of density in an ensemble of superfluid particles propagate macroscopically in an oscillatory fashion. \\
    However, rarely the full phenomenology of this and many others physical settings can be completely captured considering their waves as free, i.e. assuming they obey the D'alambert equation. The motion in time of a single harmonic of a given wavelenght is not oblivious to the motion of all others, different waves interact and influence each other. This complexity is analyzed through the tools of nonlinear physics, as the true governing equations of a great deal of physical systems are in fact nonlinear. 
    \hl{
    cita peierls 1922 \\
    usa articolo nazarenko(desktop) e intro storica suo libro per storia tramite articoli\\
    Reference history of the subject (Hassleman, Zakharov, etc)\\
    motiva fisicamente lo studio delle soluzioni di Kz\\
    parallelo con storia turbo hydro\\
    fai un botto di esempi fisici con tante belle foto (parti da mail Onorato e libro zakh per cercarli)\\
    }
%\newpage
%\thispagestyle{intro2}
%provaprova

