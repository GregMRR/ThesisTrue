% !TEX root = ../../thesis.tex


\newpage
\vphantom{}
\section{A one dimensional model for Wave Turbulence}
    The Majda-McLaughlin-Tabak (MMT) family of models was introduced in \cite{Majda1997} to test the prediction of weak turbulence theory. The idea is to have a 1D system manifesting
    the main phenomenology of nonlinear physics while at the same time being relatively easy to numerically simulate with a fine grid. Its study was then continued in
    \cite{Cai2001} and \cite{Zakharov2001}, and resulted in a better understanding of how coherent phenomena (like solitons or quasisolitons) influence the applicability 
    of wave turbulence theory. This two parameters family of models has since been a traditional benchmark on which to test new techniques and ideas in the world of turbulence. \\
    In this chapter we briefly introduce the MMT family of differential equations and discuss its salient features and associated WKE, mainly inspired by the account in 
    \cite{ZAKHAROV2004}. We then 
    move to the analysis of its higher order kinetic equation, proving that energy and wave action are still conserved quantities. Unexpectedly  we find
    ulterior divergences, neither UV or IR, due to nonlocal interactions. We prove that for all nontrivial members of the MMT family of
    models this divergences exist, and must be accounted for. Our original idea was to perform numerical simulations of the higher order WKE, to check the form of eventual
    out of equilibrium stationary states. This newly found diverging integrals didn't allow us to do so. Nontheless we introduce the basic workings of WavKinS,
    a newly developed solver for WKEs introduced in \cite{Giorgio1} and \cite{Giorgio2} by Krstulovic. We end the chapter by discussing a few simulations obtained 
    through it. \\  
    \subsection{The MMT model and its wave kinetic equation}

    In coordinate space the model consists in the two parameters family of differential equations
    \begin{equation}
        i\pt \Psi = \left|\px\right|^\alpha \Psi + \lambda \left|\px\right|^{\frac{\beta}{4}}
        \left( \left|\left|\px\right|^{\frac{\beta}{4}}\Psi\right|\left|\px\right|^{\frac{\beta}{4}}\Psi \right).
        \label{MMTPOS}
    \end{equation}
    where $\Psi = \Psi(x,t)$ is a complex function, $\lambda = \pm 1$ and $\alpha, \beta \in \mathbb{R}$. Positive $\lambda$ 
    corresponds to the defocusing case whereas negative $\lambda$ to the focusing one.\\
    We interpret the fractional derivatives in \eqref{MMTPOS} through their action in Fourier space, namely 
    \begin{equation}
        \left|\px\right|^\alpha \Psi = \frac{1}{2\pi} \int dk |k|^{\alpha} \ak e^{-ikx},
    \end{equation} 
    where $\ak = \anorm(k,t)$ is the Fourier transform of $\Psi$. A rigorous definition can be found in the appendix of \cite{ZAKHAROV2004}\\
    Time translational symmetry implies that the energy
    \begin{equation}
        \Ewav = \int dx \left( \left|\left|\px\right|^\frac{\alpha}{2}\Psi\right|^2 + \frac{1}{2}\lambda\left|\left|\px\right|^\frac{\beta}{4}\Psi\right|^4 \right)
    \end{equation}
    is conserved. Phase and translational symmetries corresponds instead to the conservation of wave action/number and momentum, with formulas
    \begin{equation}
        \Nwav = \int dx \left| \Psi \right|^2 \hspace{1cm} \textrm{M} = \frac{i}{2}\int dx \left( \Psi \px \Psi^* - \Psi^* \px \Psi \right).
    \end{equation}
    Tranforming \eqref{MMTPOS} to Fourier space we find
    \begin{equation}
        i\pt \ak = \omga_k \ak  + \int dk_1dk_2dk_3 \Tint_{k123}\aone \atwo \athreestar \delta^{k1}_{23}.
    \end{equation}
    There, to have uniform notations with the previous chapters, we defined $\omga_k = |k|^\alpha$ and $\Tint_{k123} = \lambda|kk_1k_2k_3|^\frac{\beta}{4}$. It is easy to see that
    $\Tint$ satisfies the symmetries generally required in chapter 1, allowing for the Hamiltonian treatment of the system. \\
    This system by construction only posseses a four wave interaction. In one dimension the resonance condition is
    \begin{equation}
        \begin{aligned}
            k + k_1 &= k_2 + k_3 \\
            |k|^\alpha + |k_1|^\alpha &= |k_2|^\alpha + |k_3|^\alpha.
        \end{aligned}
    \end{equation}
    In \cite{Majda1997} it was proved that the only way nontrivial solution exists\footnote{
        A nontrivial solution is a solution which does not make the collisional integral null, and thus result in an exchange of energy among modes. 
        For example $k=k_2$ and $k_1=k_3$ is always a solution, but the term 
        $\left(\frac{1}{n_k} +\frac{1}{n_1} -\frac{1}{n_2}-\frac{1}{n_3} \right)$ kills the collisional integral in that point of the resonance manifold.
    } is if $\alpha < 1$. \hl{COULD ADD PROOF} We shall then adopt $\alpha = \frac{1}{2}$ to mimic the dispersion relation of surface gravity waves, in which case $\omga = |gk|^\frac{1}{2}$ 
    with $g$ being the acceleration of gravity.\\
    By direct substitution of MMT's dispersion relation and interaction term into $\eqref{kinetic}$, the WKE to the leading order is 
    \begin{equation}
        \pt n_k = 4 \pi\int dk_1 dk_2 dk_3 \\
        |kk_1k_2k_3|^\frac{\beta}{2}n_kn_1n_2n_3
    \left(\frac{1}{n_k} + \frac{1}{n_1} - \frac{1}{n_2}- \frac{1}{n_3}  \right)
    \delta(\delomega^{k1}_{23})\delta^{k1}_{23}.
    \end{equation}
    From the discussion in chapter 1 we now that the two possible out of equilibrium solutions for this equation are 
    \begin{equation}
        \begin{aligned}
        &n_k^P \propto k^{\frac{2}{3}\beta + 1}, \\
        &n_k^Q \propto k^{\frac{2}{3}\beta + \frac{5}{6}}.
        \end{aligned}
        \label{MMTKZ}
    \end{equation}
    \hl{ADD PROOF OF LOCALITY WINDOW}\\
    Since the interaction term appears squared in the WKE, weak turbulence theory provides the same prediction in both the focusing and defocusing cases. 
    However \eqref{MMTPOS} presents different behaviours in those two cases. 
    In the focusing case stable solitonic solutions are possible, as well as wave collapses
    (\cite{Zakharov2001}). It was seen in numerical simulations of the primitive equation that, under incoherent forcing, the wave density spectrum stabilizes in a state
    with opposite wave action flux respect to the WT prediction. The energy flux sign corresponds to predictions, but only in the high $k$ limit the slope can be 
    described through a KZ state.\\
    In the defocusing case the spectrum qualitatevely corresponds to the expected one, but with a steeper slope in the inertial range. This discrepancy could be justified 
    by the presence of quasi solitons, approximate solutions of \eqref{MMTPOS} dissipating in finite time. The study of their impact on stationary states is called 
    quasisolitonic turbulence. \\
    This considerations show that the range of applicability of WT is quite narrow, and very much influenced by the choerent structures characteristic of the system. 
    Our goal in this thesis is not to discuss the validity of WT, but instead check the consistency and consequences of higher order terms into the 
    WKE of a simple system. \\
    \subsection{The higher order kinetic equation}

    By directly substituting our chosen dispersion relation and interaction term into \eqref{NLOkinetic}, the next to leading order kinetic equation for the MMT 
    model is 
    \begin{multline}
        \frac{\partial}{\partial t}n_k(t) = 4\pi \int dk_1dk_2dk_3[1 + \sigma(k,k_1,k_2,k_3)]\\
        \times |kk_1k_2k_3|^{\beta/2} n_kn_1n_2n_3 \left(\frac{1}{n_k} +\frac{1}{n_1}-
        \frac{1}{n_2}-\frac{1}{n_3}\right)\delta_{23}^{k1}\delta(\delomega^{k1}_{23}), 
        \label{wke_nlo}
    \end{multline}
    where 
    \begin{multline}
        \sigma(k,k_1,k_2,k_3) =  4 \int dk_4dk_5n_4n_5|k_4k_5|^{\beta/2} \\
        \times \left[\left( \frac{1}{n_4}+\frac{1}{n_5} \right) 
        \frac{\delta_{23}^{45}}{\sqrt{k_2}+\sqrt{k_3}-\sqrt{k_4}-\sqrt{k_5}} + 4\left( \frac{1}{n_4}-\frac{1}{n_5} \right) 
        \frac{\delta_{14}^{35}}{\sqrt{k_3}+\sqrt{k_5}-\sqrt{k_1}-\sqrt{k_4}}\right].
        \label{sigma}
    \end{multline}
    We rewrote the equation in this fashion to highlight the contributions of higher perturbative orders as modifications to the effective interaction term in the WKE.
    If we imagine the leading order WKE as accounting for all possible scatterings among waves, the $\sigma$ term here introduced accounts for secondary scatterings 
    mediated by some virtual waves $k_4$ and $k_5$. The full effect is to either suppress or enhance the interaction, resulting in a new coupling 
    $|kk_1k_2k_3|^{\beta/2} \longrightarrow [1 + \sigma(k,k_1,k_2,k_3)]|kk_1k_2k_3|^{\beta/2} $. \\

    The leading order WKE conserves energy and wave action as consequences of the symmetries of $\Tint_{k123}$. To show that even the corrected equation 
    conserves the same quantities, as it is expected, we shall just check that the same symmetries hold for $\sigma$.\\
    The symmetry under the exchange $k_2 \leftrightarrow k_3$ is built-in but not explicit, as already commented in the previous chapter. The contribution of the 
    first diagram in figure \ref{fig:quarticoneloop}is obviously symmetric and the other two contribution are mapped into each other by the exchange. It was used to make
    \eqref{NLOkinetic} more compact. \\
    Recovering it and manipulating the dummy variables, 
    $\sigma$ becomes
    \begin{multline}
      \sigma(k,k_1,k_2,k_3) =  4 \int dk_4dk_5n_4n_5|k_4k_5|^{\beta/2} 
       \left[\left( \frac{1}{n_4}+\frac{1}{n_5} \right) 
      \frac{\delta_{23}^{45}}{\sqrt{k_2}+\sqrt{k_3}-\sqrt{k_4}-\sqrt{k_5}} \right. \\
      + \left( \frac{1}{n_4}-\frac{1}{n_5} \right) \left(\frac{\delta_{14}^{35}}{\sqrt{k_3}+\sqrt{k_5}-\sqrt{k_1}-\sqrt{k_4}} - \frac{\delta_{15}^{34}}{\sqrt{k_3} 
       +\sqrt{k_4}-\sqrt{k_1}-\sqrt{k_5}}  \right. \\
      \left. \left. +\frac{\delta_{14}^{25}}{\sqrt{k_2}+\sqrt{k_5}-\sqrt{k_1}-\sqrt{k_4}} -\frac{\delta_{15}^{24}}{\sqrt{k_2}
      +\sqrt{k_4}-\sqrt{k_1}-\sqrt{k_5}}\right)\right].
    \end{multline}
    Under the exchange $(k \text{,} k_1) \leftrightarrow (k_2 \text{,} k_3)$  we obtain
    \begin{multline*}
      \sigma(k,k_1,k_2,k_3) =  4 \int_{-\infty}^{+\infty}dk_4dk_5n_4n_5|k_4k_5|^{\beta/2} 
       \left[\left( \frac{1}{n_4}+\frac{1}{n_5} \right) 
      \frac{\delta_{1k}^{45}}{\sqrt{k}+\sqrt{k_1}-\sqrt{k_4}-\sqrt{k_5}} \right. \\
      + 4\left( \frac{1}{n_4}-\frac{1}{n_5} \right) 
      \left(\frac{\delta_{34}^{15}}{\sqrt{k_1}+\sqrt{k_5}-\sqrt{k_3}-\sqrt{k_4}} - \frac{\delta_{35}^{14}}{\sqrt{k_1} +\sqrt{k_4}-\sqrt{k_3}-\sqrt{k_5}}  \right. \\
      \left. \left. +\frac{\delta_{34}^{k5}}{\sqrt{k}+\sqrt{k_5}-\sqrt{k_3}-\sqrt{k_4}} -\frac{\delta_{35}^{k4}}{\sqrt{k}+\sqrt{k_4}-\sqrt{k_3}-\sqrt{k_5}}\right)\right].
    \end{multline*}
    Imposing $\omega + \omega_1 = \omega_2 +\omega_3$ and $k + k_1 = k_2 + k_3$ results in
    \begin{multline*}
      \sigma(k,k_1,k_2,k_3) =  44 \int dk_4dk_5n_4n_5|k_4k_5|^{\beta/2} 
       \left[\left( \frac{1}{n_4}+\frac{1}{n_5} \right) 
      \frac{\delta_{1k}^{45}}{\sqrt{k_2}+\sqrt{k_3}-\sqrt{k_4}-\sqrt{k_5}} \right. \\
      + \left( \frac{1}{n_4}-\frac{1}{n_5} \right) 
      \left(\frac{\delta_{34}^{15}}{\sqrt{k_1}+\sqrt{k_5}-\sqrt{k_3}-\sqrt{k_4}} - \frac{\delta_{35}^{14}}{\sqrt{k_1} +\sqrt{k_4}-\sqrt{k_3}-\sqrt{k_5}}  \right. \\
      \left. \left. +\frac{\delta_{25}^{14}}{\sqrt{k_2}+\sqrt{k_5}-\sqrt{k_1}-\sqrt{k_4}} -\frac{\delta_{24}^{15}}{\sqrt{k_2}+\sqrt{k_4}-\sqrt{k_1}-\sqrt{k_5}}\right)\right],
    \end{multline*}
    showing that $\sigma(k,k_1,k_2,k_3) = \sigma(k_2,k_3,k,k_1) = \sigma(k,k_1,k_3,k_2)$ and thus that $E$ and $N$ are conserved\footnote{
        The two symmetries we proved togheter imply the one under exchange of $k$ and $k_1$}.\\ 
    There are two integrations in \eqref{wke_nlo}, an external one (three integration measures with two deltas) and an internal one (two measures with one delta). 
    To assess the validity of a possible stationary solution one should check the convergence of both integrals in all possible regimes of their arguments. One could
    slightly deform one of solutions \eqref{mMMTKZ}, and assess qualitatively the contribution of $\sigma$ to the KZ spectrum. This line of thought 
    was explored for the NLS equation in \cite{Rosenhaus:2025mgj}. We shall not discuss this points here however. Even before thinking about specific solutions 
    the integrals have to converge for a generic $n_k$, allowing for non pathological time evolution. We shall not concern ourselves with IR and UV divergencies,
    as those are easily healed through cutoffs. Those cutoffs may be inspired by the underlying physics of the system under consideration, for example 
    finite size imposes a natural UV cutoff and granularity of the medium imposes an IR one. They could be even imposed by the interactions of the system with 
    the environment, a dissipative interaction in the low $k$ region denies the transfer of energy or wave action across it domain. \\
    We shall instead focus our attention on the denominators of $\sigma$. Their integration should be intended in the principal value sense, but this only allows to 
    assign a value to usually indeterminate integrals. We have to explicitely check that this value is finite. We shall turn to this problem for the remaining part 
    of the section.\\

    MMT nlo, qunatità ancora conservate e discussioni sulla sua convergenza e sulla natura delle correzioni\\
    Discuti convergenza del frequency shift esplicitamente.
    divergenza e dimostrazione dell'esistenza della divergenza per $\alpha < 1$\\
    \subsection{Numerical simulations through WavKinS}

    Spiega struttura codice, conto per esplicitare delta function, mostra simulazioni per vari tipi di cascate con i loro parametri e discutile
    \subsection{Future work}
    
    Possibilità per curare divergenza, possibile introduzione di frequency renomalization (che diverge sempre)

